\documentclass[15pt]{article}
\usepackage{graphicx,fullpage,blindtext,amsmath,amssymb,listings,pgfplots,pgfplotstable}
\graphicspath{ {images/} }
\usepackage[utf8]{inputenc}
\usepackage[absolute]{textpos}
\setlength{\TPHorizModule}{1cm}
\setlength{\TPVertModule}{1cm}
\pgfplotsset{compat=1.7}
\usepackage{tikz}
\title{\emph{2020-2021\\Game Theory and Its Applications\\CS 9042}}
\author{
Arghya Bandyopadhyay 20CS4103
}
\date{27th March 2021}
\begin{document}
\begin{textblock}{5}(14,1)
\noindent\Large \[\frac{M.Tech./Even/2020-21}{\textbf{\emph{Assignment}}}\]
\end{textblock}
\maketitle
\begin{enumerate}
\item
Find out Mixed-Nash equilibrium for Rock-Paper-Scissor game

\textbf{\emph{Answer: }}
This is another problem of algebra, only now each player has 3 strategies. Let $p1_{rock}$ be the probability that
Player 1 plays Rock, $p1_{paper}$ be the probability that Player 1 plays Paper, and $p1_{scissors}$ be the probability
that Player 1 plays Scissors. Let $p2_{rock}$, $p2_{paper}$, and $p2_{scissors}$ be the respective probabilities for Player 2.
Player 1's goal is to make Player 2 indifferent among his pure strategies. We know that if Player 1 uses his
mixed strategy and Player 2 ALWAYS chooses Rock, then Player 2 will receive 0 with probability $p1_{rock}$, 1
with probability $p1_{paper}$, and 1 with probability $p1_{scissors}$. If Player 2 ALWAYS chooses Paper, then Player
2 will receive 1 with probability $p1_{rock}$, 0 with probability $p1_{paper}$, and 1 with probability $p1_{scissors}$. If
Player 2 ALWAYS chooses Scissors, then Player 2 will receive 1 with probability $p1_{rock}$, 1 with probability
$p1_{paper}$, and 0 with probability p1scissors. We can say that the expected value for Player 2 of playing each
of these strategies is then:

\begin{equation} \label{eq1}
\begin{split}
\mathbb{E}_2[Rock] & = 0*p1_{rock}+(-1)*p1_{paper}+1*p1_{scissors}\\
\mathbb{E}_2[Paper] & = 1*p1_{rock}+0*p1_{paper}+(-1)*p1_{scissors}\\
\mathbb{E}_2[Scissors] & = (-1)*p1_{rock}+1*p1_{paper}+0*p1_{scissors}\\
\end{split}
\end{equation}

We now have 3 unknowns - $p1_{rock}$, $p1_{paper}$, and $p1_{scissors}$. It must be that Player 2 has:

\begin{equation} \label{eq2}
\begin{split}
\mathbb{E}_2[Rock] & =\mathbb{E}_2[Paper]\\
\mathbb{E}_2[Paper] & =\mathbb{E}_2[Scissors]\\
\end{split}
\end{equation}

By transitivity, this gives that E2 [Rock] = E2 [Scissors]. However, there are only 2 equations and 3
unknowns. The third equation is that probabilities must sum to 1, so that our 3 equations are now:

\begin{equation} \label{eq3}
\begin{split}
0*p1_{rock}+(-1)*p1_{paper}+1*p1_{scissors} 
& =1*p1_{rock}+0*p1_{paper}+(-1)*p1_{scissors}\\
1*p1_{rock}+0*p1_{paper}+(-1)*p1_{scissors}
& =(-1)*p1_{rock}+1*p1_{paper}+0*p1_{scissors}\\
p1_{rock}+p1_{paper}+p1_{scissors}&=1\\
\end{split}
\end{equation}

We can now solve the 3 equations for $p1_{rock}$, $p1_{paper}$, and $p1_{scissors}$. Rewrite $p1_{scissors}$ = 1-$p1_{rock} - p1_{paper}$
and substitute into the first two equations. We get:

\begin{equation} \label{eq4}
\begin{split}
(-1)*p1_{paper}+1*(1-p1_{rock} - p1_{paper}) 
& =1*p1_{rock}+(-1)*(1-p1_{rock} - p1_{paper})\\
1*p1_{rock}+(-1)*(1-p1_{rock} - p1_{paper})
& =(-1)*p1_{rock}+1*p1_{paper}\\
\end{split}
\end{equation}

Now it is just a simple matter of solving the system.

\begin{equation} \label{eq5}
\begin{split}
-p1_{paper}+1-p1_{rock} - p1_{paper} 
& =p1_{rock}-1+p1_{rock} +p1_{paper}\\
p1_{rock}-1+p1_{rock} +p1_{paper}
& =-p1_{rock}+p1_{paper}\\
\end{split}
\end{equation}

Simplifying:

\begin{equation} \label{eq6}
\begin{split}
-3*p1_{paper}+2 
& =3*p1_{rock}\\
3*p1_{rock}-1&=0\\
\end{split}
\end{equation}

We have $p1_{rock}= \frac{1}{3}$
from the last equation. Substituting that into the first equation we get:

\begin{equation} \label{eq7}
\begin{split}
-3*p1_{paper}+2 
& =3*\frac{1}{3}\\
\end{split}
\end{equation}

Solving for $p1_{paper}$ gives $p1_{paper}= \frac{1}{3}$. Now, using $p1_{scissors}$ = 1-$p1_{rock} - p1_{paper}$ we find that 

\begin{equation} \label{eq8}
\begin{split}
p1_{scissors}
& =\frac{1}{3}\\
\end{split}
\end{equation}


So if Player 1 plays Rock $\frac{1}{3}$ of the time, Paper $\frac{1}{3}$ of the time, and Scissors $\frac{1}{3}$ of the time this will make Player 2 indifferent over his pure strategies. The expected value for Player 2 of playing Rock is 0, of playing Paper
is 0, and of playing Scissors is 0. We can also check some mixed strategies for Player 2. If Player 2 plays Rock 50\% of the time and Paper 50\% of the time his expected value of playing that strategy is 0. If Player 2 plays Rock 50\% of the time, Paper 25\% of the time, and Scissors 25\% of the time his expected value is 0. This is what is required when finding a MSNE.

Now, we have only found the probabilities for Player 1. We know that all strategies (pure or mixed) by Player 2 provide the same expected value when Player 1 chooses Rock, Paper, and Scissors $\frac{1}{3}$ of the time each. Can Player 2 then just choose any strategy? No, because not any strategy will make Player 1 indifferent over his strategies. We then have to and $p2_{rock}$, $p2_{paper}$, $p2_{scissors}$, using the same methodology that we just used to find $p1_{rock}$, , $p1_{paper}$, $p2_{scissors}$. Luckily, in the Rock, Paper, Scissors game the two players
have symmetric payoffs and strategies, so that the probabilities for Player 2 that make Player 1 indifferent
between his strategies are $p2_{rock}=\frac{1}{3}, p2_{paper}=\frac{1}{3}, p2_{scissors}=\frac{1}{3},$. The mixed strategy Nash equilibrium for this game (and the only Nash equilibrium of this game) is that Player 1 chooses Rock, Paper, and Scissors each with $\frac{1}{3}$ probability, and Player 2 chooses Rock, Paper, and Scissors each with $\frac{1}{3}$ probability. Note that the expected value for both players of playing this game is 0. 

Now, suppose that Player 2 uses a different strategy, like $p2_{rock}=\frac{1}{3}, p2_{paper}=\frac{1}{2}, p2_{scissors}=\frac{1}{6},$.
Should Player 1 respond with the exact same strategy? NO. If Player 1 uses the exact same strategy as
Player 2 then his expected value of using that strategy is also 0. Player 1 can do BETTER than 0 if he uses
a strategy like always choose Scissors. If Player 1 uses this strategy against $p2_{rock}=\frac{1}{3}, p2_{paper}=\frac{1}{2}, p2_{scissors}=\frac{1}{6},$.
then Player 1 will have an expected value of $\frac{1}{6}$
because he will earn (1) with $\frac{1}{3}$ probability
(when Player 2 picks Rock), 1 with $\frac{1}{2}$ probability (when Player 2 picks Paper) and 0 with $\frac{1}{6}$ probability
(when Player 2 chooses Scissors). Of course, if Player 1 always chooses Scissors then Player 2 would always choose Rock. But then Player 1 would always choose Paper. And the cycle would go on. The only time it stops is when $p1_{rock}=\frac{1}{3}, p1_{paper}=\frac{1}{3}, p1_{scissors}=\frac{1}{3}, p2_{rock}=\frac{1}{3}, p2_{paper}=\frac{1}{3}, p2_{scissors}=\frac{1}{3},$

\item Devise one example which helps to establish that Ranked Choice Voting is not strategy proof. Manipulation should be done in a way that our preferred candidate gets more favoured match-ups in later round and alternatives should be eliminated early

\textbf{\emph{Answer: }}True Preference Result:
\begin{center}
\begin{tabular}{ |c|c|c|c| } 
 \hline
 \multicolumn{4}{|c|}{Iteration 1} \\ 
 \hline
 Choice & Voter 1 2 & Voter 3 4 & Voter 5  \\ [0.5ex] 
 \hline
 $1^{st}$ & a & b & c \\ 
 \hline
 $2^{nd}$ & d & a & d \\ 
 \hline
 $3^{rd}$ & c & d & b \\ 
 \hline
 $4^{th}$ & b & c & a \\
 \hline
 \multicolumn{4}{|c|}{Iteration 2} \\ 
 \hline
 Choice & Voter 1 2 & Voter 3 4 & Voter 5  \\ [0.5ex] 
 \hline
 $1^{st}$ & a & b & c \\ 
 \hline
 $2^{nd}$ & c & a & b \\ 
 \hline
 $3^{rd}$ & b & c & a \\
 \hline
 \multicolumn{4}{|c|}{Iteration 3} \\ 
 \hline
 Choice & Voter 1 2 & Voter 3 4 & Voter 5  \\ [0.5ex] 
 \hline
 $1^{st}$ & a & b & b \\ 
 \hline
 $2^{nd}$ & b & a & a \\ [1ex] 
 \hline
\end{tabular}
\end{center}
Here B wins.

Results after manipulation (When voter 1 and voter 2 switch to their $2^{nd}$ favourite choice):
\begin{center}
\begin{tabular}{ |c|c|c|c| } 
 \hline
 \multicolumn{4}{|c|}{Iteration 1} \\ 
 \hline
 Choice & Voter 1 2 & Voter 3 4 & Voter 5  \\ [0.5ex] 
 \hline
 $1^{st}$ & d & b & c \\ 
 \hline
 $2^{nd}$ & a & a & d \\ 
 \hline
 $3^{rd}$ & c & d & b \\ 
 \hline
 $4^{th}$ & b & c & a \\
 \hline
 \multicolumn{4}{|c|}{Iteration 2} \\ 
 \hline
 Choice & Voter 1 2 & Voter 3 4 & Voter 5  \\ [0.5ex] 
 \hline
 $1^{st}$ & d & b & c \\ 
 \hline
 $2^{nd}$ & c & d & d \\ 
 \hline
 $3^{rd}$ & b & c & b \\
 \hline
 \multicolumn{4}{|c|}{Iteration 3} \\ 
 \hline
 Choice & Voter 1 2 & Voter 3 4 & Voter 5  \\ [0.5ex] 
 \hline
 $1^{st}$ & d & b & d \\ 
 \hline
 $2^{nd}$ & b & d & b \\ [1ex] 
 \hline
\end{tabular}
\end{center}
Here D wins.

\item Try to find any algorithm which will give both-side optimal result in stable matching problem. Find whether this is possible or not?

\textbf{\emph{Answer: }} Suppose a situation :

\begin{center}
\begin{tabular}{ |c|c|c|c| } 
 \hline
 \multicolumn{4}{|c|}{Preference List} \\ 
 \hline
 Boy & Boy's Preferences & Girl & Girl's Preferences  \\ [0.5ex] 
 \hline
 $B_1$ & $G_2>G_1$ & $G_1$ & $B_1>B_2$ \\ 
 \hline
 $B_2$ & $G_1>G_2$ & $G_2$ & $B_2>B_1$ \\ [1ex] 
 \hline
\end{tabular}
\end{center}

Two possible stable solution are:
Solution 1, gives priority to boys and Solution 2, gives priority to girls.

\begin{center}
\begin{tabular}{ |c|c| } 
 \hline
 \multicolumn{2}{|c|}{Solution 1} \\ 
 \hline
 Boy & Girl  \\ [0.5ex] 
 \hline
 $B_1$ & $G_2$ \\ 
 \hline
 $B_2$ & $G_1$ \\ [1ex] 
 \hline
\end{tabular}
\begin{tabular}{ |c|c| } 
 \hline
 \multicolumn{2}{|c|}{Solution 2} \\ 
 \hline
 Boy & Girl  \\ [0.5ex] 
 \hline
 $B_1$ & $G_1$ \\ 
 \hline
 $B_2$ & $G_2$\\ [1ex] 
 \hline
\end{tabular}
\end{center}

No, matter what there was no instance to be found which provide optimal solution to both side because they are acting as contrary to one another.

\textbf{So devising Algorithm is not possible till now.}

\item 
Is Indian election system using plurality voting rule?

\textbf{\emph{Answer: }} Since, in order to win one need votes more than any other single candidate, majority vote is not required, \textbf{Yes}, India’s election System follows plurality voting rule. Plurality voting is used for local and/or national elections in 43 of the 193 countries that are members of the United Nations. It is particularly prevalent in the United Kingdom, the United States, Canada and India. 

\item 
Devise one scenario with a preference list in which one is incrementally improving under deferred acceptance algorithm(i.e. even if one is accepted at the time being and if later on a more preferred choice comes up He/She will be better off).

\textbf{\emph{Answer: }}

\begin{center}
\begin{tabular}{ |c|c|c|c|c| } 
 \hline
 \multicolumn{5}{|c|}{Preference List} \\ 
 \hline
 Student & Student's Preferences & Hospital & Capacity & Hospital's Preferences  \\ [0.5ex] 
 \hline
 $A$ & $H_1 H_2 H_3$ & $H_1$ & 2 & $C A B D$ \\ 
 \hline
 $B$ & $H_1 H_2 H_3$ & $H_2$ & 1 & $D A B C$ \\ 
 \hline
 $C$ & $H_2 H_1 H_3$ & $H_3$ & 1 & $A B C D$ \\ 
 \hline
 $D$ & $H_2 H_1 H_3$ &  & & \\ [1ex] 
 \hline
\end{tabular}
\end{center}

Allotment to the student will be done accordingly:

\begin{center}
\begin{tabular}{ |c|c|c|c| } 
 \hline
 \multicolumn{4}{|c|}{Iteration 1 allotment list} \\ 
 \hline
 Student & Hospital Allotted & Hospital & Student Allotted \\ [0.5ex] 
 \hline
 $A$ & $H_1$ & $H_1$ & $A B$ \\ 
 \hline
 $B$ & $H_1$ & $H_2$ & $D$ \\ 
 \hline
 $C$ & & $H_3$ & \\ 
 \hline
 $D$ & $H_2$ &  & \\ [1ex] 
 \hline
 \hline
 \multicolumn{4}{|c|}{Iteration 2 allotment list} \\ 
 \hline
 Student & Hospital Allotted & Hospital & Student Allotted \\ [0.5ex] 
 \hline
 $A$ & & $H_1$ & $B  C$ \\ 
 \hline
 $B$ & $H_1$ & $H_2$ & $D$ \\ 
 \hline
 $C$ & $H_1$ & $H_3$ & \\ 
 \hline
 $D$ & $H_2$ &  & \\ 
 \hline
 \hline
 \multicolumn{4}{|c|}{Iteration 3 allotment list} \\ 
 \hline
 Student & Hospital Allotted & Hospital & Student Allotted \\ [0.5ex] 
 \hline
 $A$ & $H_3$ & $H_1$ & $B C$ \\ 
 \hline
 $B$ & $H_1$ & $H_2$ & $D$ \\ 
 \hline
 $C$ & $H_1$ & $H_3$ & $A$\\ 
 \hline
 $D$ & $H_2$ &  & \\ [1ex] 
 \hline
\end{tabular}
\end{center}

\item Justify that the POA in the non-linear Pigou's example is unbounded as $P \to \infty$

\textbf{\emph{Answer: }} The POA is $\frac{4}{3}$ in both Braess’s Paradox and Pigou’s example — not so bad for completely unregulated behavior. Given what we currently know, the coolest thing that could be true would be if the POA of selfish routing was always at most $\frac{4}{3}$. (A rather bold guess, given that we've only looked at two examples.) The story is not so rosy in all networks, however. In the nonlinear Pigou’s example (Figure 3(b)), we replace the previous cost function $c(x) = x$ of the lower edge with the function $c(x) = x^p$ , with p large. The lower edge remains a
dominant strategy, and the equilibrium travel time remains 1. What’s changed is that the optimal solution is now much better. If we again split the traffic equally between the two links, then the average travel time tends to $\frac{1}{2}$ as $P \to \infty$ — traffic on the bottom edge gets to t nearly instantaneously. We can do even better by routing $(1\textendash\epsilon)$ traffic on the bottom
link, where $\epsilon$ tends to 0 as p tends to infinity — almost all of the traffic gets to t with travel
time $(1\textendash\epsilon)^p$ , which is close to 0 when p is sufficiently large, and the $\epsilon$ fraction of martyrs
on the upper edge contribute little to the average travel time. We conclude that the POA in the nonlinear Pigou’s example is unbounded as $P \to \infty$.

\item 
How DU is the unique, best response for each player i.e unique best strategy, verify this claim.

\textbf{\emph{Answer: }}
Payoff matrix is:

\begin{center}
\begin{tabular}{ |c|c|c| }
 \hline
 & $U$ & $DU$  \\ [0.5ex] 
 \hline
 $U$ & $(2,2)$ & $(-1,3)$ \\ 
 \hline
 $DU$ & $(3,3)$ & $(0,0)$\\ [1ex] 
 \hline
\end{tabular}
\end{center}

By analyzing the above payoff matrix, it was found that $(DU, DU )$ is a \emph{Stable Solution}. Because, in other three cases if any one player's deviates from $U \to DU$ can improve his/her playoff i.e $(DU, DU )$ is a Nash Equilibrium. So this proves the fact of stability. $(DU, DU )$ is also the unique best strategy no matter what the other player is doing as players are always inclined to maximising their payoff. Even if they agree for $(U, U )$, deviation by other players can bring loss. Hence proved.

\item 
Does proportionality implies envy-freeness Or vice-versa?

\textbf{\emph{Answer: }}\\ 
Proportionality (PR) and envy-freeness (EF) are two independent properties, but in some cases one of them may imply the other.

When all valuations are additive set functions and the entire cake is divided, the following implications hold:
\begin{enumerate}
\item With two partners, PR and EF are equivalent;
\item With three or more partners, EF implies PR but not vice versa. For example, it is possible that each of three partners receives $\frac{1}{3}$ in his subjective opinion, but in Alice's opinion, Bob's share is worth $\frac{2}{3}$.

When the valuations are only sub additive, EF still implies PR, but PR no longer implies EF even with two partners: it is possible that Alice's share is worth $\frac{1}{2}$ in her eyes, but Bob's share is worth even more. On the contrary, when the valuations are only super additive, PR still implies EF with two partners, but EF no longer implies PR even with two partners: it is possible that Alice's share is worth $\frac{1}{4}$ in her eyes, but Bob's is worth even less. Similarly, when not all cake is divided, EF no longer implies PR. The implications are summarized in the following table:
\begin{center}
\begin{tabular}{ |c|c|c| } 
 \hline
 Additive & \begin{tabular}{@{}c@{}}$EF \to PR$ \\ $PR \to EF$\end{tabular} & $EF \to PR$  \\ [0.5ex] 
 \hline
 Sub additive & $EF \to PR$ & $EF \to PR$ \\ 
 \hline
 Super additive & $PR \to EF$ & - \\ 
 \hline
 General & - & - \\ [1ex] 
 \hline
\end{tabular}
\end{center}
\end{enumerate}
\end{enumerate}
\end{document}
